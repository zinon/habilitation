\subsection*{Collaboration Roles}

\textbf{Coordinator of the SUSY electroweak multileptons analysis group of ATLAS}\newline
I am responsible for the organization of the four-lepton SUSY analysis group consisting of about 10 members.  Among the main duties of this role, I provide recommendations for data analysis, deliver processed and calibrated data, develop tools for statistical interpretation of physics results, and edit the relevant analysis papers.
\newline

\noindent
\textbf{Contact person for the tau leptons in the SUSY group of ATLAS}\newline
I am appointed as the liaison person between the SUSY analysis group and tau combined performance group. The main mandate of this role is to maintain a tight link between the SUSY analysis group the combined tau performance group by ensuring the newest analysis recommendations are available to all SUSY analyses involving taus. I am also organizing feedback from SUSY analyses on new or under-development recommendations and assisting in the review of analyses, particularly those with complex or non-standard treatment of physics objects, such as real and fake taus.
\newline

\noindent
\textbf{Tau combined performance group of ATLAS}\newline
I have the responsibility for the tau energy scale measurement using in-situ tag-and-probe techniques. More specifically, I am the main developer of the statistical model employed to define with high precision the tau energy scale and tau identification at ATLAS, and the main author of the data flattening and calibration framework.

\subsubsection*{Personal contributions}
I played a leading role in all aspects of this physics search. In particular, I was responsible for designing the full analysis strategy, providing the analysis software framework, developing sophisticated background estimate techniques, reconstructing calibrated data, preparing the Monte Carlo simulation data samples, and performing the statistical combination and interpretation. 

In addition, I was coordinating the efforts of the corresponding analysis group. I managed a working group of about ten active members, which administered many tasks such as analysis software development, data sample preparation, data-driven measurements of the main background processes, and long-term optimization studies for understanding the irreducible backgrounds. 

Being intimately involved in all these activities, I also provided direction and guidance for future improvements for signal region definitions. I ensured that the analyzers were collaborating and communicating ideas, while
continuously providing constructive feedback and suggestions, such that these ideas could be realized and used for other ATLAS analyses involving similar final states. I oversaw all recommendations for systematic uncertainties on the electron, muon and tau identification efficiency, energy scale, and misidentification probabilities, to ensure their quality and robustness.

Finally, I coordinated the supporting documentation to ensure that the aspects of the analysis were well described and I had the role of chief editor of the public paper describing the final results of this search. Being the editor of this paper, I was able to put together a concise public document under extreme time pressure, and I was able to incorporate a plethora of comments from the collaboration while ensuring that the document was written in a very
professional manner.

---------------------
More specifically, this paper presents the results from a search for SUSY in events with four or more charged leptons (electrons, muons and taus) by exploiting a data sample corresponding to 36~$\mathrm{fb}^{-1}$ of pp collisions 
delivered by the LHC at  $\sqrt{s}=13$ TeV and collected by ATLAS during the years 2015 and 2016.
%Four-lepton signal regions with up to two hadronically decaying taus are designed to target a range of supersymmetric scenarios that can be either enriched in or depleted of events involving the production and decay of a Z boson. 
Data yields are found to be consistent with SM expectations and results are used to set upper limits on the event yields from processes beyond the SM. In absence of significant excess of data over the predicted background events, exclusion limits are set at the 95\% confidence level in simplified models of General Gauge Mediated (GGM) SUSY and in RPV simplified models with decays of the LSP to charged leptons and limits being placed on wino, slepton and gluino masses, respectively.


%Data yields were found to be consistent with SM expectations and results were used to set upper limits on the event yields from processes beyond the SM. Exclusion limits were set at the 95\% confidence level in simplified models of general-gauge-mediated supersymmetry, where Higgsino masses are excluded up to $\sim300$ GeV. 
%In RPV simplified models with decays of the lightest supersymmetric particle to charged leptons, lower limits of about 1.5, 1.0, and 2.3 TeV are placed on wino, slepton and gluino masses, respectively.
